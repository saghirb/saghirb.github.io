\documentclass[]{book}
\usepackage{lmodern}
\usepackage{amssymb,amsmath}
\usepackage{ifxetex,ifluatex}
\usepackage{fixltx2e} % provides \textsubscript
\ifnum 0\ifxetex 1\fi\ifluatex 1\fi=0 % if pdftex
  \usepackage[T1]{fontenc}
  \usepackage[utf8]{inputenc}
\else % if luatex or xelatex
  \ifxetex
    \usepackage{mathspec}
  \else
    \usepackage{fontspec}
  \fi
  \defaultfontfeatures{Ligatures=TeX,Scale=MatchLowercase}
\fi
% use upquote if available, for straight quotes in verbatim environments
\IfFileExists{upquote.sty}{\usepackage{upquote}}{}
% use microtype if available
\IfFileExists{microtype.sty}{%
\usepackage{microtype}
\UseMicrotypeSet[protrusion]{basicmath} % disable protrusion for tt fonts
}{}
\usepackage[margin=1in]{geometry}
\usepackage{hyperref}
\hypersetup{unicode=true,
            pdftitle={Saghir's Notebook Wiki},
            pdfborder={0 0 0},
            breaklinks=true}
\urlstyle{same}  % don't use monospace font for urls
\usepackage{natbib}
\bibliographystyle{apalike}
\usepackage{color}
\usepackage{fancyvrb}
\newcommand{\VerbBar}{|}
\newcommand{\VERB}{\Verb[commandchars=\\\{\}]}
\DefineVerbatimEnvironment{Highlighting}{Verbatim}{commandchars=\\\{\}}
% Add ',fontsize=\small' for more characters per line
\usepackage{framed}
\definecolor{shadecolor}{RGB}{248,248,248}
\newenvironment{Shaded}{\begin{snugshade}}{\end{snugshade}}
\newcommand{\KeywordTok}[1]{\textcolor[rgb]{0.13,0.29,0.53}{\textbf{#1}}}
\newcommand{\DataTypeTok}[1]{\textcolor[rgb]{0.13,0.29,0.53}{#1}}
\newcommand{\DecValTok}[1]{\textcolor[rgb]{0.00,0.00,0.81}{#1}}
\newcommand{\BaseNTok}[1]{\textcolor[rgb]{0.00,0.00,0.81}{#1}}
\newcommand{\FloatTok}[1]{\textcolor[rgb]{0.00,0.00,0.81}{#1}}
\newcommand{\ConstantTok}[1]{\textcolor[rgb]{0.00,0.00,0.00}{#1}}
\newcommand{\CharTok}[1]{\textcolor[rgb]{0.31,0.60,0.02}{#1}}
\newcommand{\SpecialCharTok}[1]{\textcolor[rgb]{0.00,0.00,0.00}{#1}}
\newcommand{\StringTok}[1]{\textcolor[rgb]{0.31,0.60,0.02}{#1}}
\newcommand{\VerbatimStringTok}[1]{\textcolor[rgb]{0.31,0.60,0.02}{#1}}
\newcommand{\SpecialStringTok}[1]{\textcolor[rgb]{0.31,0.60,0.02}{#1}}
\newcommand{\ImportTok}[1]{#1}
\newcommand{\CommentTok}[1]{\textcolor[rgb]{0.56,0.35,0.01}{\textit{#1}}}
\newcommand{\DocumentationTok}[1]{\textcolor[rgb]{0.56,0.35,0.01}{\textbf{\textit{#1}}}}
\newcommand{\AnnotationTok}[1]{\textcolor[rgb]{0.56,0.35,0.01}{\textbf{\textit{#1}}}}
\newcommand{\CommentVarTok}[1]{\textcolor[rgb]{0.56,0.35,0.01}{\textbf{\textit{#1}}}}
\newcommand{\OtherTok}[1]{\textcolor[rgb]{0.56,0.35,0.01}{#1}}
\newcommand{\FunctionTok}[1]{\textcolor[rgb]{0.00,0.00,0.00}{#1}}
\newcommand{\VariableTok}[1]{\textcolor[rgb]{0.00,0.00,0.00}{#1}}
\newcommand{\ControlFlowTok}[1]{\textcolor[rgb]{0.13,0.29,0.53}{\textbf{#1}}}
\newcommand{\OperatorTok}[1]{\textcolor[rgb]{0.81,0.36,0.00}{\textbf{#1}}}
\newcommand{\BuiltInTok}[1]{#1}
\newcommand{\ExtensionTok}[1]{#1}
\newcommand{\PreprocessorTok}[1]{\textcolor[rgb]{0.56,0.35,0.01}{\textit{#1}}}
\newcommand{\AttributeTok}[1]{\textcolor[rgb]{0.77,0.63,0.00}{#1}}
\newcommand{\RegionMarkerTok}[1]{#1}
\newcommand{\InformationTok}[1]{\textcolor[rgb]{0.56,0.35,0.01}{\textbf{\textit{#1}}}}
\newcommand{\WarningTok}[1]{\textcolor[rgb]{0.56,0.35,0.01}{\textbf{\textit{#1}}}}
\newcommand{\AlertTok}[1]{\textcolor[rgb]{0.94,0.16,0.16}{#1}}
\newcommand{\ErrorTok}[1]{\textcolor[rgb]{0.64,0.00,0.00}{\textbf{#1}}}
\newcommand{\NormalTok}[1]{#1}
\usepackage{longtable,booktabs}
\usepackage{graphicx,grffile}
\makeatletter
\def\maxwidth{\ifdim\Gin@nat@width>\linewidth\linewidth\else\Gin@nat@width\fi}
\def\maxheight{\ifdim\Gin@nat@height>\textheight\textheight\else\Gin@nat@height\fi}
\makeatother
% Scale images if necessary, so that they will not overflow the page
% margins by default, and it is still possible to overwrite the defaults
% using explicit options in \includegraphics[width, height, ...]{}
\setkeys{Gin}{width=\maxwidth,height=\maxheight,keepaspectratio}
\IfFileExists{parskip.sty}{%
\usepackage{parskip}
}{% else
\setlength{\parindent}{0pt}
\setlength{\parskip}{6pt plus 2pt minus 1pt}
}
\setlength{\emergencystretch}{3em}  % prevent overfull lines
\providecommand{\tightlist}{%
  \setlength{\itemsep}{0pt}\setlength{\parskip}{0pt}}
\setcounter{secnumdepth}{5}
% Redefines (sub)paragraphs to behave more like sections
\ifx\paragraph\undefined\else
\let\oldparagraph\paragraph
\renewcommand{\paragraph}[1]{\oldparagraph{#1}\mbox{}}
\fi
\ifx\subparagraph\undefined\else
\let\oldsubparagraph\subparagraph
\renewcommand{\subparagraph}[1]{\oldsubparagraph{#1}\mbox{}}
\fi

%%% Use protect on footnotes to avoid problems with footnotes in titles
\let\rmarkdownfootnote\footnote%
\def\footnote{\protect\rmarkdownfootnote}

%%% Change title format to be more compact
\usepackage{titling}

% Create subtitle command for use in maketitle
\newcommand{\subtitle}[1]{
  \posttitle{
    \begin{center}\large#1\end{center}
    }
}

\setlength{\droptitle}{-2em}

  \title{Saghir's Notebook `Wiki'}
    \pretitle{\vspace{\droptitle}\centering\huge}
  \posttitle{\par}
    \author{}
    \preauthor{}\postauthor{}
      \predate{\centering\large\emph}
  \postdate{\par}
    \date{Version: 20 September 2018 (11:33)}

\usepackage{booktabs}
\usepackage{amsthm}
\makeatletter
\def\thm@space@setup{%
  \thm@preskip=8pt plus 2pt minus 4pt
  \thm@postskip=\thm@preskip
}
\makeatother

\usepackage{amsthm}
\newtheorem{theorem}{Theorem}[chapter]
\newtheorem{lemma}{Lemma}[chapter]
\theoremstyle{definition}
\newtheorem{definition}{Definition}[chapter]
\newtheorem{corollary}{Corollary}[chapter]
\newtheorem{proposition}{Proposition}[chapter]
\theoremstyle{definition}
\newtheorem{example}{Example}[chapter]
\theoremstyle{definition}
\newtheorem{exercise}{Exercise}[chapter]
\theoremstyle{remark}
\newtheorem*{remark}{Remark}
\newtheorem*{solution}{Solution}
\begin{document}
\maketitle

{
\setcounter{tocdepth}{1}
\tableofcontents
}
\begin{Shaded}
\begin{Highlighting}[]
\KeywordTok{library}\NormalTok{(tidyverse)}
\KeywordTok{library}\NormalTok{(lubridate)}
\KeywordTok{library}\NormalTok{(readODS)}
\KeywordTok{library}\NormalTok{(knitr)}
\KeywordTok{library}\NormalTok{(kableExtra)}
\end{Highlighting}
\end{Shaded}

\chapter{Administrative Info}\label{administrative-info}

\textbf{Personal}\footnote{Copy \& Paste friendly versions}

\begin{verbatim}
Saghir Bashir
Address Line 1
Address Line 2
Address Line 3
Country

Tel: +xx 123 456 789
GSM: +xx 987 654 321

Email: example@example.com
Website: http://dsup.org
SS Number: AAA-NNNNNNNNNN
...
\end{verbatim}

\textbf{Professional}

\begin{verbatim}
Organisation Name
Address Line 1
Address Line 2
Address Line 3
Country


Tel: +xx 321 456 789
GSM: +xx 789 654 321

Email: work@example.com
Website: http://ilustat.com

Emergency Contact: <HR Depart?>
Emergency Phone: <Number>
Emergency Email: <sick@example.com>
\end{verbatim}

\begin{itemize}
\tightlist
\item
  \href{https://speakerdeck.com/saghirb}{Speaker Deck (saghirb)}
\item
  \href{https://medium.com/@saghirb}{Medium (saghirb)}
\item
  \href{http://ilustat.com/}{ilustat}
\item
  \href{http://dsup.org/}{Data Science Unplugged (dsup)}
\end{itemize}

\chapter{Bookmarks}\label{bookmarks}

\textbf{News}

\begin{itemize}
\tightlist
\item
  \href{https://www.bbc.com/news}{BBC News}
\item
  \href{https://theguardian.com}{Guardian}
\item
  \href{https://newint.org/}{New Internationlist}
\item
  \href{https://www.wikitribune.com/}{Wiki Tribune}
\item
  \href{http://wttr.in/London}{wttr.in/London}
\end{itemize}

\textbf{Geek News}

\begin{itemize}
\tightlist
\item
  \href{https://news.ycombinator.com/}{Hacker News}
\item
  \href{https://rweekly.org}{R weekly}
\item
  \href{https://www.statslife.org.uk/}{Stats Life}
\end{itemize}

\textbf{Login free Twitter :)}

\begin{itemize}
\tightlist
\item
  \href{https://twitter.com/hashtag/rstats?src=hash}{\#rstats}
\item
  \href{https://twitter.com/d_spiegel}{David Spiegelhalter}
\item
  \href{https://twitter.com/hadleywickham}{Hadley Wickham}
\item
  \href{https://twitter.com/ilustat}{ilustat}
\item
  \href{https://twitter.com/MattDowle}{Matt Dowle}
\item
  \href{https://twitter.com/xieyihui}{Yihui Xie}
\end{itemize}

\textbf{Check Accounts}

\begin{itemize}
\tightlist
\item
  \href{https://www.example.com/}{My bank}
\item
  \href{https://www.example.com/}{My other bank}
\item
  \href{https://www.example.com/}{Professional Society}
\end{itemize}

\textbf{IT Security}

\begin{itemize}
\tightlist
\item
  \href{https://www.schneier.com/}{Bruce Schneier}
\item
  \href{https://krebsonsecurity.com/}{Krebs on Security}
\end{itemize}

\textbf{Linux/Unix}

\begin{itemize}
\tightlist
\item
  \href{http://seankross.com/the-unix-workbench/}{Book: The Unix
  Workbench}
\item
  \href{https://www.cyberciti.biz/}{nixCraft}
\end{itemize}

\textbf{DS - Docs}

\begin{itemize}
\tightlist
\item
  \href{https://bookdown.org/yihui/bookdown/}{Bookdown book -- Xie
  Yihui}
\item
  \href{http://happygitwithr.com/}{Happy With Git R}
\item
  \href{https://github.com/rstudio/RStartHere}{R Start Here}
\item
  \href{https://bookdown.org/yihui/rmarkdown/}{Rmarkdown}
\end{itemize}

\textbf{Fun}

\begin{itemize}
\tightlist
\item
  \href{https://www.theonion.com/}{The Onion}
\item
  \href{https://xkcd.org/}{xkcd}
\end{itemize}

\textbf{Food}

\begin{itemize}
\tightlist
\item
  \href{https://www.bbcgoodfood.com}{BBC Good Food}
\end{itemize}

\textbf{MacOS}

\begin{itemize}
\tightlist
\item
  \href{https://brew.sh/}{Brew}
\end{itemize}

\chapter{Useful Links}\label{Links}

\texttt{\textless{}Section\ simplified\textgreater{}}

\section{Medium}\label{medium}

\begin{itemize}
\tightlist
\item
  \href{https://medium.com/@hugorcf}{Hugo Ferreira}
\end{itemize}

\section{To Read}\label{to-read}

\begin{itemize}
\tightlist
\item
  \href{https://community.rstudio.com/tags/bookdown-contest}{Bookdown
  Contest}
\item
  \href{https://nullprogram.com/blog/2018/09/06/}{Brute Force Incognito
  Browsing}
\item
  \href{https://thebaffler.com/salvos/privatizing-poverty-phillips-fein}{Privatizing
  Povery}
\end{itemize}

\section{To Watch}\label{to-watch}

\begin{itemize}
\tightlist
\item
  \href{https://www.ted.com/talks/onora_o_neill_what_we_don_t_understand_about_trust}{Onora
  O'Neill: What we don't understand about trust}
\end{itemize}

\section{Archive -- Read}\label{archive-read}

\begin{itemize}
\tightlist
\item
  \href{https://www-cs-faculty.stanford.edu/~knuth/email.html}{Donald
  Knuth versus Email}
\item
  \href{https://theoutline.com/post/5495/how-to-beat-linked-in-the-game}{How
  to beat LinkedIn: The Game}
\item
  \href{https://www.ibm.com/developerworks/aix/library/au-unixtext/index.html}{Introduction
  to text manipulation on UNIX-based systems}
\item
  \href{http://www.strikemag.org/bullshit-jobs/}{On the Phenomenon of
  Bullshit Jobs: A Work Rant by David Graeber}
\item
  \href{https://austinkleon.com/2018/08/30/reading-with-a-pencil/}{Reading
  with a Pencil}
\end{itemize}

\section{Archive -- Video}\label{archive-video}

\begin{itemize}
\tightlist
\item
  \href{https://www.ted.com/talks/onora_o_neill_what_we_don_t_understand_about_trust}{Onora
  O'Neill: What we don't understand about trust}
\end{itemize}

\chapter{Life Admin}\label{life-admin}

This contains information that I usually have to look up in an emergency
or to feed the daily bureaucracy we encounter
\texttt{\textless{}sigh!\textgreater{}}.

\section{Government}\label{government}

\textbf{Tax Authority}

\begin{verbatim}
<censored>
  - Tax reference number / Social Security
  - Address
  - Customer service telephone number
  - Website
  - Secure login site
\end{verbatim}

\textbf{Residency related}

\begin{verbatim}
<censored>
  - Residency / ID card number
  - Address
  - Customer service telephone number
  - Website
  - Secure login website
\end{verbatim}

\textbf{Healthcare}

\begin{verbatim}
<censored>
Government
  - Healthcare number
  - Address
  - Customer service telephone number
  - Website
  - Secure login website
  
Insurance
  - Insurance reference number
  - Address
  - Customer service telephone number
  - Website
  - Secure login website
\end{verbatim}

\section{Utilities}\label{utilities}

\textbf{Electricity}

\begin{verbatim}
<censored>

Information to include:
  - Customer reference number
  - Address
  - Customer service telephone number
  - Website
  - Secure login site
  - Emergency contact info
\end{verbatim}

\textbf{Gas}

\begin{verbatim}
See Electricity for template.
\end{verbatim}

\textbf{Water}

\begin{verbatim}
See Electricity for template.
\end{verbatim}

\textbf{Internet, telephone, cable}

\begin{verbatim}
<censored>
\end{verbatim}

\textbf{Mobile phone}

\begin{verbatim}
See Electricity for template.
\end{verbatim}

\section{Financial}\label{financial}

\textbf{Bank Name}

\begin{verbatim}
<censored>

Information to include:
  - Account numbers
  - Bank Address
  - Customer service telephone number
  - Website
  - Secure login site
  - Emergency contact info
\end{verbatim}

\textbf{Credit cards}

\begin{verbatim}
<censored>

Information to include:
  - Card numbers
  - Provider
  - Customer service telephone number
  - Website
  - Secure login site
  - Emergency contact info (for cancellations)
\end{verbatim}

\textbf{Pensions}

\begin{verbatim}
<censored>

Information to include:
  - Pension Plan Number
  - Provider
  - Address
  - Customer service telephone number
  - Website
  - Secure login site
\end{verbatim}

\section{Address Change}\label{address-change}

Inform the following of any change in address:

\begin{itemize}
\tightlist
\item
  Tax authority
\item
  Government residency
\item
  Banks
\item
  Pension providers
\item
  Family member 1
\item
  ..
\item
  Family member n
\end{itemize}

\chapter{Address Book Plus}\label{address-book-plus}

\section{Birthdays \& Anniversaries}\label{birthdays-anniversaries}

Type

Month

Date

Name

Birthday

Jan

10-1-1993

David Bartlett FAKE INFO

12-1-NA

Aimee Anthony FAKE INFO

13-1-NA

Shirley Brady FAKE INFO

19-1-1987

Kasey Ellis FAKE INFO

Feb

3-2-1988

Sandy Little FAKE INFO

12-2-1990

Lloyd Hodge FAKE INFO

23-2-1985

Marylou Church FAKE INFO

Mar

13-3-1982

Alisa Guerrero FAKE INFO

14-3-1966

Landon Munoz FAKE INFO

18-3-1962

Angelo Woods FAKE INFO

26-3-1987

Rosemarie Contreras FAKE INFO

30-3-1971

Noelle Leonard FAKE INFO

Apr

8-4-1973

Angelo Woods FAKE INFO

May

1-5-1986

Isabella Foreman FAKE INFO

4-5-1982

Claudio Allison FAKE INFO

6-5-1979

Celia Morrison FAKE INFO

14-5-1976

Laverne Hopper FAKE INFO

24-5-NA

Kurtis Mcleod FAKE INFO

Jun

1-6-1969

Elnora Pitts FAKE INFO

16-6-1966

Wilson Bonner FAKE INFO

Jul

5-7-1974

Sally Rollins FAKE INFO

6-7-1984

Taylor Avery FAKE INFO

14-7-1995

Traci Ballard FAKE INFO

25-7-1962

Ashley Everett FAKE INFO

Aug

14-8-1961

August Stuart FAKE INFO

21-8-1986

Vera Flores FAKE INFO

23-8-NA

Georgina Valencia FAKE INFO

Sep

27-9-1970

Claudia Baxter FAKE INFO

28-9-1993

Luz Carroll FAKE INFO

28-9-1984

Tod Burton FAKE INFO

Oct

2-10-1984

Bessie Rosales FAKE INFO

17-10-1965

Jarred Rojas FAKE INFO

24-10-1977

Alisa Guerrero FAKE INFO

28-10-1971

Aimee Anthony FAKE INFO

Nov

10-11-1977

Terrence Chambers FAKE INFO

19-11-1988

Chasity Cox FAKE INFO

29-11-NA

Anita Alexander FAKE INFO

Dec

13-12-1974

Marla Franks FAKE INFO

22-12-1973

Bessie Kinney FAKE INFO

Wedding

Jul

25-7-1995

Jarred Rojas FAKE INFO

Oct

8-10-2008

Marylou Church FAKE INFO

\section{People Info}\label{people-info}

Name

Telephone

Address

Aimee Anthony FAKE INFO

+1 708 xxx xxxx

1002 Hamlet St Alton, IL 62002 USA

Alisa Guerrero FAKE INFO

+1 708 xxx xxxx

65 Plymth Terr Aurora, IL 60507 USA

Angelo Woods FAKE INFO

+1 512 xxx xxxx

257 Willow Rd Austin, TX 78710 USA

Anita Alexander FAKE INFO

+1 409 xxx xxxx

548 Main St Galveston, TX 77553 USA

Ashley Everett FAKE INFO

+1 919 xxx xxxx

174 Orange West Burlington, NC 27215 USA

August Stuart FAKE INFO

+1 409 xxx xxxx

961 Tellfly St Galveston, TX 77553 USA

Bessie Kinney FAKE INFO

+1 708 xxx xxxx

404 Dorwin Rd Aurora, IL 60507 USA

Bessie Rosales FAKE INFO

+1 313 xxx xxxx

397 Willow Rd Detroit, MI 48233 USA

Celia Morrison FAKE INFO

+1 414 xxx xxxx

439 Erming Ln Milwaukee, WI 53201 USA

Chasity Cox FAKE INFO

+1 201 xxx xxxx

952 Brandy Run Passaic, NJ 07055 USA

Claudia Baxter FAKE INFO

+1 316 xxx xxxx

474 Columbus Dr Emporia, KS 66801 USA

Claudio Allison FAKE INFO

+1 814 xxx xxxx

563 Lake Dr Erie, PA 16515 USA

David Bartlett FAKE INFO

+1 515 xxx xxxx

407 West Street Terr Ames, IA 50010 USA

Elnora Pitts FAKE INFO

+1 512 xxx xxxx

996 Brighton St Austin, TX 78710 USA

Georgina Valencia FAKE INFO

+1 602 xxx xxxx

244 Cedarwood Ln Phoenix, AZ 85026 USA

Isabella Foreman FAKE INFO

+1 708 xxx xxxx

768 Second St Alton, IL 62002 USA

Jarred Rojas FAKE INFO

+1 518 xxx xxxx

727 First St Albany, NY 12212 USA

Kasey Ellis FAKE INFO

+1 206 xxx xxxx

818 Main St Vancouver, WA 98661 USA

Kurtis Mcleod -- FAKE INFO

+1 313 xxx xxxx

507 New First Rd Detroit, MI 48233 USA

Landon Munoz FAKE INFO

+1 513 xxx xxxx

895 Plymth Terr Hamilton, OH 45012 USA

Laverne Hopper FAKE INFO

+1 903 xxx xxxx

43 Freeton Blvd Irving, TX 75061 USA

Lloyd Hodge FAKE INFO

+1 708 xxx xxxx

611 Freeton Blvd Alton, IL 62002 USA

Luz Carroll FAKE INFO

+1 708 xxx xxxx

665 Dorwin Rd Aurora, IL 60507 USA

Marla Franks FAKE INFO

+1 316 xxx xxxx

820 Tellfly St Wichita, KS 67276 USA

Marylou Church FAKE INFO

+1 404 xxx xxxx

458 Brighton St Athens, GA 30601 USA

Noelle Leonard FAKE INFO

+1 615 xxx xxxx

413 Maple Ln Knoxville, TN 37901 USA

Rosemarie Contreras FAKE INFO

+1 305 xxx xxxx

164 Stonehedge Blvd Miami, FL 33152 USA

RStudio Inc

+1 844-448-1212

250 Northern Ave, Boston, MA 02210

Sally Rollins FAKE INFO

+1 414 xxx xxxx

197 Spring County Blvd Appleton, WI 54911 USA

Sandy Little FAKE INFO

+1 510 xxx xxxx

285 Second St Berkeley, CA 94704 USA

Sherlock Holmes

+44xxx xxx xxxx

221B Baker Street London United Kingdom

Shirley Brady FAKE INFO

+1 316 xxx xxxx

212 Beley Rd Wichita, KS 67276 USA

Taylor Avery FAKE INFO

+1 408 xxx xxxx

217 Main St Sunnyvale, CA 94086 USA

Terrence Chambers FAKE INFO

+1 219 xxx xxxx

318 Plymth Terr Gary, IN 46401 USA

Tod Burton FAKE INFO

+1 409 xxx xxxx

932 Stonehedge Blvd Galveston, TX 77553 USA

Traci Ballard FAKE INFO

+1 513 xxx xxxx

942 Kennel Ln Hamilton, OH 45012 USA

Vera Flores FAKE INFO

+1 513 xxx xxxx

47 Third St Dayton, OH 45401 USA

Wilson Bonner FAKE INFO

+1 315 xxx xxxx

767 Spring County Blvd Auburn, NY 13021 USA

\chapter{People Facts}\label{people-facts}

Useful to know things when socialising with family or friends.

Name

Fact

Angelo Woods FAKE INFO

Gluten intolerance

Angelo Woods FAKE INFO

Vegetarian

Celia Morrison FAKE INFO

Allergic to pet hair

Claudio Allison FAKE INFO

Lactose \& tomato intolerance

Elnora Pitts FAKE INFO

Vegetarian

Isabella Foreman FAKE INFO

Vegetarian

Kurtis Mcleod FAKE INFO

Fear of dogs

Landon Munoz FAKE INFO

Garlic intolerance

Laverne Hopper FAKE INFO

Vegan

Noelle Leonard FAKE INFO

No strawberries

Rosemarie Contreras FAKE INFO

Not a vegetarian food fan

Wilson Bonner FAKE INFO

Meat only. No fruit or vegetables except potatoes.

\chapter{Data Science}\label{data-science}

\section{Statistics}\label{statistics}

\texttt{\textless{}Section\ Cut\textgreater{}}

\section{Rstats}\label{rstats}

\begin{itemize}
\tightlist
\item
  \href{https://r-mageddon.netlify.com/post/writing-an-r-package-from-scratch/}{Writing
  a package by Tomas Westlake}
\end{itemize}

\textbf{tidyverse}

\begin{itemize}
\tightlist
\item
  \href{https://github.com/suzanbaert/SatRdaysAmsterdam18_dplyr}{Suazan
  Baert dplyr Presentation Amsterdam 2018}
\item
  \href{https://github.com/jennybc/row-oriented-workflows}{Jenny Bryan's
  Row-oriented workflows in R with the tidyverse}
\end{itemize}

\textbf{purrr Package}

\begin{itemize}
\tightlist
\item
  \href{https://github.com/jenniferthompson/RLadiesIntroToPurrr}{Introduction
  to purrr}
\item
  \href{https://jennybc.github.io/purrr-tutorial/}{Jenny Bryan's purrr
  Tutorial}
\end{itemize}

\section{Git \& Github}\label{git-github}

\textbf{Some useful references:}

\begin{itemize}
\tightlist
\item
  \href{http://happygitwithr.com/}{Happy Git and GitHub for the useR} --
  Jenny Bryan
\item
  \href{Happy\%20Git\%20and\%20GitHub\%20for\%20the\%20useR}{Git guides}
  -- Mara Averick
\item
  \href{http://r-pkgs.had.co.nz/git.html}{Git and GitHub} -- R Packages
  by Hadley Wickham
\item
  \href{https://rmhogervorst.nl/cleancode/blog/2016/03/01/content/post/2016-03-01-version-control-start/}{Version
  control with Git} -- Roel Hogervorst
\end{itemize}

\textbf{Working with a remote upstream libary}

\begin{verbatim}
  git remote -v
  git fetch upstream
  git checkout master
  git merge upstream/master
\end{verbatim}

\section{Books}\label{books}

\begin{itemize}
\tightlist
\item
  \href{https://bookdown.org/yihui/rmarkdown/}{R Markdown: The
  Definitive Guide}
\item
  \href{https://bookdown.org/yihui/bookdown/}{Bookdown}
\item
  \href{https://bookdown.org/yihui/blogdown/}{blogdown: Creating
  Websites with R Markdown}
\item
  \href{http://r4ds.had.co.nz/}{R fo Data Science}
\item
  \href{https://socviz.co/}{Data Visualization}
\item
  \href{Handling\%20Strings\%20with\%20R}{Handling Strings with R}
\item
  \href{https://bookdown.org/csgillespie/efficientR/}{Efficient R
  programming}
\item
  \href{https://adv-r.hadley.nz/}{Advanced R}
\item
  \href{https://www.datascienceatthecommandline.com/}{Data Science at
  the Command Line}
\end{itemize}

\section{R code}\label{r-code}

\textbf{tidyverse stuff}

\begin{itemize}
\tightlist
\item
  \href{https://github.com/saghirb/Getting-Started-in-R}{Getting Started
  in R}
\end{itemize}

\textbf{flexdashboard}

\begin{itemize}
\tightlist
\item
  \href{https://github.com/saghirb/Chick_Weight_Flexdashboard}{Chick
  weight data flexdashboard}
\end{itemize}

\textbf{R Shiny}

\begin{itemize}
\tightlist
\item
  \href{https://github.com/saghirb/Chick_Weight_Basic_App}{Chick weight
  data app}
\item
  \href{https://github.com/saghirb/Nutri_PT}{Portuguese Nutrition App}
\end{itemize}

\texttt{\textless{}cut\textgreater{}}

\chapter{Linux}\label{linux}

\section{Command line - bash}\label{command-line---bash}

\textbf{Finding duplicates files}
\texttt{fdupes\ -rSm\ .\ \textgreater{}\ duplicates.txt}

\textbf{DANGEROUS - Removing duplicates automatically - DANGEROUS}
\texttt{fdupes\ -rdN}

\textbf{Recursively shred/erase}
\texttt{find\ \textless{}dir\textgreater{}\ -depth\ -type\ f\ -exec\ shred\ -v\ -n\ 1\ -z\ -u\ \{\}\ \textbackslash{};}

\textbf{Find largest file in a directory and subdirectories}

\begin{verbatim}
find <dir> -printf '%s %p\n'| sort -nr | head -10`
find . -type f -size +10000k -exec ls -lh {} \;
\end{verbatim}

\textbf{Disk usage by directory (human readable form)}
\texttt{du\ -h\ -\/-max-depth=1\ \textbar{}\ sort\ -h}

\textbf{Rename files using string matching}
\texttt{rename\ \textquotesingle{}s/JPG/jpg/g\textquotesingle{}\ *.JPG}

\textbf{Rename recursively through sub directories}
\texttt{find\ .\ -depth\ -exec\ rename\ \textquotesingle{}s/JPG/jpg/g\textquotesingle{}\ \{\}\ +}

\textbf{Randomly order the lines in a file} \texttt{shuf\ -n\ 1\ \$FILE}

\textbf{Unique file exensions}
\texttt{find\ *\ \textbar{}\ awk\ -F\ .\ \{\textquotesingle{}print\ \$2\textquotesingle{}\}\ \textbar{}\ sort\ \textbar{}\ uniq\ -c}

\textbf{Randomly select words of any length}
\texttt{cat\ /usr/share/dict/british-english\ \textbar{}\ shuf\ -n\ 10}

\textbf{Randomnly select words of length 6}

\begin{verbatim}
grep -E '^[[:alpha:]]{6}$' /usr/share/dict/british-english | shuf -n 10
\end{verbatim}

\textbf{Translate words}
\texttt{dict\ -d\ \textless{}dict\textgreater{}\ \textless{}word\textgreater{}}

\textbf{Available dictionaries} \texttt{dict\ -I} or \texttt{dict\ -D}

\textbf{Weather} \texttt{curl\ wttr.in/london}

\textbf{Extract filename and extension in bash}

\begin{Shaded}
\begin{Highlighting}[]
\VariableTok{FILE=}\StringTok{"example.tar.gz"}
\BuiltInTok{echo} \StringTok{"}\VariableTok{$\{FILE%%}\NormalTok{.*}\VariableTok{\}}\StringTok{"}
\BuiltInTok{echo} \StringTok{"}\VariableTok{$\{FILE%}\NormalTok{.*}\VariableTok{\}}\StringTok{"}
\BuiltInTok{echo} \StringTok{"}\VariableTok{$\{FILE#}\NormalTok{*.}\VariableTok{\}}\StringTok{"}
\BuiltInTok{echo} \StringTok{"}\VariableTok{$\{FILE##}\NormalTok{*.}\VariableTok{\}}\StringTok{"}
\end{Highlighting}
\end{Shaded}

\begin{verbatim}
## example
## example.tar
## tar.gz
## gz
\end{verbatim}

\section{Backups}\label{backups}

\textbf{tar}

\begin{verbatim}
tar -cvpzf backup.tar.gz --exclude=/backup.tar.gz --one-file-system / 
sudo tar --exclude='/home/user/???' --exclude='/home/user/???' cvpf destination.tar  /home/user
# Reset numeric owner (useful for sharing between systems)
tar --numeric-owner --group=1000 --owner=1000 ... 
\end{verbatim}

\textbf{rsync}

\begin{verbatim}
rsync -rtvu --exclude "Pictures" --exclude "Downloads" --exclude "Music" /home/user/ /media/.../
rsync -rtvu --delete source_folder/ destination_folder/
\end{verbatim}

\textbf{Command line Weather} \texttt{curl\ wttr.in/lisbon}

\textbf{Extract links from web page}
\texttt{lynx\ -listonly\ -nonumbers\ -dump\ http://www.example.com}

\section{PDF \& LaTeX}\label{pdf-latex}

\textbf{Split a pdf file}
\texttt{pdftk\ file.pdf\ cat\ 12-15\ output\ outfile\_p12-15.pdf}

\textbf{Merge pdf files}
\texttt{pdftk\ file1.pdf\ file2.pdf\ cat\ output\ out.pdf}

\textbf{PDF Total number of pages}
\texttt{exiftool\ -T\ -filename\ -PageCount\ -s3\ -ext\ pdf\ \textless{}dir\textgreater{}}

\textbf{Converting a text file to PDF}
\texttt{pandoc\ file.txt\ -o\ file.pdf}

\textbf{Converting a text file to HTML}
\texttt{pandoc\ file.txt\ -o\ file.html}

\textbf{Converting images to PDF}
\texttt{convert\ -page\ A4\ img1.jpg\ -append\ img2.jpg\ -append\ img.pdf}

\textbf{Stamping/overlaying a PDF file}
\texttt{pdftk\ file.pdf\ multistamp\ stamp.pdf\ output\ Stamped.pdf}

\textbf{PDF to PNG or JPG (images) handles multiple pages}

\begin{verbatim}
pdftoppm -png file.pdf outFileStem
pdftoppm -jpeg -gray file.pdf outFileStem
\end{verbatim}

\textbf{Clean file names (-n for dryrun)}
\texttt{detox\ -r\ -v\ \textless{}dir\textgreater{}} e.g.
\texttt{detox\ -r\ -v\ \$(pwd)}

\textbf{Unencryting PDFs}

\begin{verbatim}
gs -q -dNOPAUSE -dBATCH -sDEVICE=pdfwrite -sOutputFile=unencrypted.pdf \
  -c .setpdfwrite -f encrypted.pdf
\end{verbatim}

\textbf{Script to create handouts for course notes}

\begin{verbatim}
#!/bin/sh

for pdffile in slides1.pdf slides2.pdf slides3.pdf     
do  
  pdfjam --batch --nup 2x2 --scale 0.8 --frame true --a4paper --landscape \
         --delta "0.7cm 0.7cm" --suffix "4up" --outfile . $pdffile
done     

pdfjoin --outfile Handouts.pdf slides1-4up.pdf slides2-4up.pdf slides3-4up.pdf
\end{verbatim}

\section{Image \& Video Processing}\label{image-video-processing}

\textbf{Before Sharing Photos}

\begin{verbatim}
# exiftool part explained below.
exiftool -r '-FileName<CreateDate' -d 'DESCRIPTION_%Y%m%d-%H%M%S%%-c.%%le' .
mogrify -strip *.jpg
\end{verbatim}

\textbf{Remove EXIF info and resize}
\texttt{mogrify\ -strip\ -resize\ 25\%\ *.JPG}

\textbf{Rotate photos by 90 degrees} \texttt{mogrify\ -rotate\ 90\ *JPG}

\textbf{Convert from jpg to PDF} \texttt{convert\ *.JPG\ pics.pdf}

\textbf{Rename files using EXIF information} \textbf{Rename with date
and time from EXIF info}

** Rename image files per EXIF data**

\emph{Source:}
(\url{https://superuser.com/questions/205417/sort-and-rename-images-by-date-in-exif-info})

\begin{quote}
On Linux you can use the command exiftool. For some reason the online
manual does not contain the ``RENAMING EXAMPLES'' section which gave me
the essential hint. For JPG only files the following command invocation
should do the job
\end{quote}

\begin{verbatim}
exiftool -r '-FileName<CreateDate' -d 'XXXX_%Y%m%d-%H%M%S%%-c.%%le' .
# To arrange in folders by year
exiftool -r '-FileName<CreateDate' -d '/home/user/Pictures/Keep/%Y/Orig_%Y%m%d-%H%M%S%%-c.%%le' <dir>
\end{verbatim}

\textbf{Explanation:}

\begin{quote}
-r is for recursion
\end{quote}

\begin{quote}
`-FileName\textless{}CreateDate' tells exiftool to rename the file
accordingly to its EXIF tag CreateDate (you can try others like
ModifyDate though)
\end{quote}

\begin{quote}
-d `\%Y\%m\%d-\%H\%M\%S\%\%-c.\%\%le' tells how to build the filename
string from the date source ``CreateDate'' (the ``\%-c'' will append a
counter in case of file collisions, the ``\%le'' stands for ``lower
cased file extension'')
\end{quote}

\begin{quote}
Note: I used ``-FileName\textless{}\ldots{}'' here for renaming the
files and moving it to another folder within one step. The manual points
out that you have to use the ``-Directory\textless{}\ldots{}'' syntax
for folder operations. It worked for me this way though.
\end{quote}

\textbf{Video file info to CSV}

\begin{verbatim}
exiftool -r -imagewidth -imageheight -duration -filesize -csv * > temp.csv
\end{verbatim}

\emph{Source:}
(\url{https://superuser.com/questions/205417/sort-and-rename-images-by-date-in-exif-info})
\emph{Short link:} \url{http://bit.ly/1HhFcfO}

\textbf{Remove camera related info}
\texttt{exiftool\ -MakerNotes:all=\textquotesingle{}\textquotesingle{}\ *.JPG}

\textbf{Convert webm to mp4}
\texttt{ffmpeg\ -i\ "v1.webm"\ -qscale\ 0\ "v1.mp4"}

\textbf{Convert webm to mp4 recursively} Two methods

\begin{verbatim}
find -name "*.webm" -exec ffmpeg -i {} -qscale 0 {}.mp4 \;
\end{verbatim}

\begin{verbatim}
for i in *.webm ; do 
    ffmpeg -i "$i" -qscale 0 $(basename "${i/.webm}").mp4
    sleep 60 
done
\end{verbatim}

\section{Security}\label{security}

\textbf{Checksum}

\emph{Source:}
(\url{https://askubuntu.com/questions/318530/generate-md5-checksum-for-all-files-in-a-directory})

\begin{quote}
great checksum creation/verification program is rhash. It creates even
SFV compatible files, and checks them too. It supports md4, md5, sha1,
sha512, crc32 and many many other. Moreover it can do recursive creation
(-r option) like md5deep or sha1deep. Last but not least you can format
the output of the checksum file; for example:
\end{quote}

\begin{verbatim}
rhash -r --md5 -p '%h,%p /home/\n'
\end{verbatim}

\begin{quote}
outputs a CSV file including the full path of files.
\end{quote}

\textbf{Encrpyted compressed 7zip archive}

\begin{verbatim}
7z a -t7z -m0=lzma2 -mx=9 -mfb=64 -md=32m -ms=on -mhe=on -p'eat_my_shorts' archive.7z dir1
\end{verbatim}

\begin{longtable}[]{@{}rl@{}}
\toprule
option & Details\tabularnewline
\midrule
\endhead
a & Add (dir1 to archive.7z)\tabularnewline
-t7z & Use a 7z archive\tabularnewline
-m0=lzma2 & Use lzma2 method\tabularnewline
-mx=9 & Use the `9' level of compression = Ultra\tabularnewline
-mfb=64 & Use number of fast bytes for LZMA = 64\tabularnewline
-md=32m & Use a dictionary size = 32 megabytes\tabularnewline
-ms=on & Solid archive = on\tabularnewline
-mhe=on & 7z format only : enables or disables archive header
encryption\tabularnewline
-p\{Password\} & Add a password\tabularnewline
\bottomrule
\end{longtable}

\section{System \& Hardware}\label{system-hardware}

\textbf{Useful commands (see man pages)} \texttt{dmidecode},
\texttt{lshw}, \texttt{inxi} (prefer \texttt{inxi})

\textbf{Identify hardware model and other info}

\begin{verbatim}
sudo dmidecode -t1
inxi -M
inxi -bxx
lshw -short
\end{verbatim}

\textbf{Identify installed Linux Distro} \texttt{lsb\_release\ -a}

\section{Debian}\label{debian}

\textbf{Debian Packages of R Software:} Backports info at
(\url{https://cran.r-project.org/bin/linux/debian/})

\textbf{Version installed} \texttt{lsb\_release\ -a}

\textbf{All currently installed packages.}

\begin{verbatim}
dpkg --get-selections
dpkg-query -l
# Just extracting package names
apt-mark showmanual
dpkg-query -f '${binary:Package}\n' -W
\end{verbatim}

\textbf{Install packages from a file}
\texttt{apt-get\ install\ \$(grep\ -vE\ "\^{}\textbackslash{}s*\#"\ filename\ \ \textbar{}\ tr\ "\textbackslash{}n"\ "\ ")}

\textbf{Recreating a partition table (via root) - USE WITH CARE}

\begin{verbatim}
cfdisk /dev/sdx
mkfs -t vfat /dev/dsx1
\end{verbatim}

\section{Misc}\label{misc}

\textbf{Get MS Windows Key}
\texttt{sudo\ cat\ /sys/firmware/acpi/tables/MSDM}

\emph{Source:}
(\url{https://askubuntu.com/questions/953126/can-i-recover-my-windows-product-key-from-ubuntu})

\textbf{Calendar with history}
(\url{https://akr.am/blog/posts/today-in-history-brought-to-you-by-unix})

\textbf{Speech to text}

\emph{Source:}
(\url{https://askubuntu.com/questions/161515/speech-recognition-app-to-convert-mp3-to-text})

The software you can use is CMUSphinx. Unlike suggested in another
answer Julius is not suitable because it requires models. Models for
large vocabulary speech recognition are not available for Julius.

You can use pocketsphinx to convert audio file. Those two commands must
do the work. First you convert the file to the required format and then
you recognize it:

\begin{verbatim}
ffmpeg -i file.mp3 -ar 16000 -ac 1 file.wav
\end{verbatim}

Then run pocketsphinx (results in \texttt{file.txt})

\begin{verbatim}
pocketsphinx_continuous -infile file.wav 2> pocketsphinx.log > file.txt
\end{verbatim}

\textbf{Text to Speech}

\textbf{pico2wave} The voice is more human sounding.

Save the following as \texttt{tts.sh}

\begin{verbatim}
#!/bin/bash
pico2wave -l=en-GB -w=/tmp/test.wav "$1"
aplay /tmp/test.wav
rm /tmp/test.wav
\end{verbatim}

Test it with: ./tts.sh ``Hello World! Where are you living? Since
when.''

You can also use a file:

\begin{verbatim}
pico2wave -l=en-GB -w=test.wav "$(cat file.txt)"
\end{verbatim}

\textbf{espeak} A robotic voice with more language choices.

\begin{verbatim}
espeak --stdout "this is a test" | paplay
echo "these are my notes" > text.txt
espeak --stdout -f text.txt > text.wav
paplay text.wav
\end{verbatim}

\textbf{Correcting file names}

The detox utility renames files to make them easier to work with. It
removes spaces and other such annoyances. It'll also translate or
cleanup Latin-1 (ISO 8859-1) characters encoded in 8-bit ASCII, Unicode
characters encoded in UTF-8, and CGI escaped characters.

\begin{verbatim}
detox -s iso8859_1 -r -v -n /tmp/new_files
\end{verbatim}

Will run the sequence iso8859\_1 recursively, listing any changes,
without changing anything, on the files of /tmp/new\_files.

\chapter{Docker}\label{docker}

\textbf{Remove containers}

\begin{verbatim}
docker ps
docker ps -a
docker rm $(docker ps -qa --no-trunc --filter "status=exited")
\end{verbatim}

\textbf{How to cleanup (unused) resources (use carefully)}

\emph{Source:}
(\url{https://gist.github.com/bastman/5b57ddb3c11942094f8d0a97d461b430})

\begin{verbatim}
dcleanup(){
    docker rm -v $(docker ps --filter status=exited -q 2>/dev/null) 2>/dev/null
    docker rmi $(docker images --filter dangling=true -q 2>/dev/null) 2>/dev/null
}
\end{verbatim}

\begin{quote}
Once in a while, you may need to cleanup resources (containers, volumes,
images, networks) \ldots{} delete volumes
\end{quote}

Source: (\url{https://github.com/chadoe/docker-cleanup-volumes})

\begin{verbatim}
docker volume rm $(docker volume ls -qf dangling=true)
docker volume ls -qf dangling=true | xargs -r docker volume rm
\end{verbatim}

\textbf{Delete networks}

\begin{verbatim}
docker network ls  
docker network ls | grep "bridge"   
docker network rm $(docker network ls | grep "bridge" | awk '/ / { print $1 }')
\end{verbatim}

\textbf{Rocker - R \& RStudio in container}

\begin{itemize}
\tightlist
\item
  See
  \href{https://github.com/rocker-org/rocker/wiki/Using-the-RStudio-image}{Rocker
  wiki} for more information.
\end{itemize}

Login: rstudio -- Password: rstudio

\begin{verbatim}
docker run -d -p 8787:8787 -v $(pwd):/home/rstudio rocker/r-base
docker run -d -p 8787:8787 -v $(pwd):/home/rstudio rocker/tidyverse
docker run -d -p 8787:8787 -v $(pwd):/home/rstudio rocker/verse
\end{verbatim}

\textbf{Apache}

\begin{verbatim}
docker run -p 80:80 -dit --name http24 -v "$PWD":/usr/local/apache2/htdocs/ httpd:2.4
\end{verbatim}

\bibliography{book.bib,packages.bib}


\end{document}
